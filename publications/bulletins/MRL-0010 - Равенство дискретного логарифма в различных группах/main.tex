\documentclass{mrl}
\usepackage[utf8x]{inputenc}
\usepackage[T2A]{fontenc}
\usepackage[english, russian]{babel}

\title{Равенство дискретного логарифма в различных группах}
\authors{Саранг Ноезер (Sarang Noether)\footnote{\texttt{sarang.noether@protonmail.com}}}
\affiliations{Исследовательская лаборатория Monero (Monero Research Lab)}
\date{04 Декабря 2018}

\newcommand{\hg}{\operatorname{H}_\mathbb{G}}
\newcommand{\hh}{\operatorname{H}_\mathbb{H}}
\newcommand{\zp}{\mathbb{Z}_p}
\newcommand{\zq}{\mathbb{Z}_q}

\type{ТЕХНИЧЕСКАЯ ЗАПИСКА}
\ident{MRL-0010}

\begin{document}

\begin{center}
{\bfАннотация}
\end{center}

В данной технической записке содержится описание алгоритма, обеспечивающего доказательство знания дискретного логарифма в различных группах. Схема выражает общее значение в виде скалярного представления битов и использует набор кольцевых подписей для доказательства того, что значение каждого бита действительно и одинаково (вплоть до полной эквивалентности) в обеих скалярных группах.

\section{Обозначения}
Нами используется $\mathbb{Z}_n$ для короткого обозначения группы $\mathbb{Z}/n\mathbb{Z}$. Допустим, $\mathbb{G}$ и $\mathbb{H}$ являются группами первого порядка, в которых задача доказательства дискретного логарифма является \newline сложной: например, \texttt{secp256k1} или $l$-подгруппой \texttt{curve25519}. Допустим, $G,G' \in \mathbb{G}$ и $H,H' \in \mathbb{H}$ будет являются генераторами соответствующих групп. Предположим, $|G| = p$ и $|H| = q$. Допустим, $\hg: \{0,1\}^* \to \zp$ и $\hh: \{0,1\}^* \to \zq$ являются криптографическими хеш-функциями.

Без потери общности, допустим, что $p \leq q$. Выберем такое значение $x \in \mathbb{Z}$, чтобы $0 \leq x < p$. Рассмотрим естественные проекции $\mathbb{Z} \to \zp$ и $\mathbb{Z} \to \zq$ с таким ограничением области определения, и увидим взаимно-однозначное соответствие между элементами $\zp$ и ограничением $\zq$. Учитывая это, нам нужно доказать, что только при наличии значений $xG'$ и $xH'$ (и, при необходимости, других элементов доказательства) дискретный логарифм обеих будет представлением одного и того же числа. В частности, при этом мы не хотим раскрывать $x$ верификатору.

Так как значимая связь между двумя группами отсутствует, наш подход состоит в разложении $x$ на биты. При этом каждый бит будет рассматриваться как скалярная величина как в $\zp$, так и в $\zq$ в рамках нашей эквивалентности, а обязательства будут генерироваться для каждого бита в обеих группах. Для каждого бита нами будет построена кольцевая подпись Шнорра, что продемонстрирует, что обязательство по биту является действительным и имеет одно и то же значение в каждой группе.

Данный метод был изначально предложен Эндрю Поэлстра (Andrew Poelstra).
\section{Алгоритм}
\subsection{Доказывающая сторона}
У нас есть число $0 \leq x < p$, выраженное в битах: $$x = \sum_{i=0}^{n-1} b_i2^i$$ Следует отметить, что из-за эквивалентности, о которой говорилось выше, каждый $b_i$ по необходимости может рассматриваться в качестве элемента либо группы $\zp$, либо группы $\zq$, в результате чего $x$ будет представлен в каждой группе.
Для каждого $i \in [0,n-2]$ генерируем произвольные блайндеры $r_i \in \zp$ и $s_i \in \zq$. Для $i = n-1$ устанавливаем блайндеры $$r_{n-1} = (2^{n-1})^{-1}\sum_{i=0}^{n-2} r_i2^i \in \zp$$ и $$s_{n-1} = (2^{n-1})^{-1}\sum_{i=0}^{n-2} s_i2^i \in \zq$$ чтобы гарантировать, что $\sum_{i=0}^{n-1} r_i2^i = \sum_{i=1}^{n-1} s_i2^i = 0$.

Для каждого $i \in [0,n-1]$ используем блайндеры, чтобы вычислить два обязательства Педерсена:
\begin{eqnarray*}
C_i^G &:=& b_iG' + r_iG \in \mathbb{G} \\
C_i^H &:=& b_iH' + s_iH \in \mathbb{H}
\end{eqnarray*}
Из-за такой конструкции взвешенными суммами обязательств в соответствующих группах будут \newline $\sum_{i=0}^{n-1} 2^iC_i^G = xG'$ и $\sum_{i=0}^{n-1} 2^iC_i^H = xH'$.

Затем строим кольцевую подпись, по каждому биту, чтобы продемонстрировать, что значение будет либо $0$, либо $1$, и это значение будет одним и тем же (вплоть до полной эквивалентности) в обеих группах. В частности, для каждого $i \in [0,n-1]$ нами рассматриваются два варианта:

\textbf{Вариант:} $b_i = 0$. Выбираем произвольные $j_i \in \zp$ и $k_i \in \zq$. Задаём
\begin{eqnarray*}
e_{1,i}^G &:=& \hg\left( C_i^G, C_i^H, j_iG, k_iH \right) \in \zp \\
e_{1,i}^H &:=& \hh\left( C_i^G, C_i^H, j_iG, k_iH \right) \in \zq
\end{eqnarray*}
и выбираем произвольные $a_{0,i} \in \zp$ и $b_{0,i} \in \zq$. Задаём
\begin{eqnarray*}
e_{0,i}^G &:=& \hg\left( C_i^G, C_i^H, a_{0,i}G - e_{1,i}^G(C_i^G-G'), b_{0,i}H - e_{1,i}^H(C_i^H-H') \right) \in \zp \\
e_{0,i}^H &:=& \hh\left( C_i^G, C_i^H, a_{0,i}G - e_{1,i}^G(C_i^G-G'), b_{0,i}H - e_{1,i}^H(C_i^H-H') \right) \in \zq
\end{eqnarray*}
а затем определяем:
\begin{eqnarray*}
a_{1,i} &:=& j_i + e_{0,i}^Gr_i \in \zp \\
b_{1,i} &:=& k_i + e_{0,i}^Hs_i \in \zq
\end{eqnarray*}

\textbf{Вариант:} $b_i = 1$. Выбираем произвольные $j_i \in \zp$ и $k_i \in \zq$. Задаём
\begin{eqnarray*}
e_{0,i}^G &:=& \hg\left( C_i^G, C_i^H, j_iG, k_iH \right) \in \zp \\
e_{0,i}^H &:=& \hh\left( C_i^G, C_i^H, j_iG, k_iH \right) \in \zq
\end{eqnarray*}
и выбираем произвольные $a_{1,i} \in \zp$ и $b_{1,i} \in \zq$. Задаём
\begin{eqnarray*}
e_{1,i}^G &:=& \hg\left( C_i^G, C_i^H, a_{1,i}G - e_{0,i}^GC_i^G, b_{1,i}H - e_{0,i}^HC_i^H \right) \in \zp \\
e_{1,i}^H &:=& \hh\left( C_i^G, C_i^H, a_{1,i}G - e_{0,i}^GC_i^G, b_{1,i}H - e_{0,i}^HC_i^H \right) \in \zq
\end{eqnarray*}
а затем определяем:
\begin{eqnarray*}
a_{0,i} &:=& j_i + e_{1,i}^Gr_i \in \zp \\
b_{0,i} &:=& k_i + e_{1,i}^Hs_i \in \zq
\end{eqnarray*}

Доказательством является кортеж $\left( xG',xH',\{C_i^G\},\{C_i^H\}, \{e_{0,i}^G\}, \{e_{0,i}^H\}, \{a_{0,i}\}, \{a_{1,i}\}, \{b_{0,i}\}, \{b_{1,i}\} \right)$.

\subsection{Верификатор}
Учитывая кортеж доказательства, нам следует убедиться в том, что обязательства по битам верно представляют доказательства дискретного логарифма. Это проверяется при помощи следующих уравнений:
\begin{eqnarray*}
\sum_{i=0}^{n-1} 2^iC_i^G &=& xG' \in \mathbb{G} \\
\sum_{i=0}^{n-1} 2^iC_i^H &=& xH' \in \mathbb{H}
\end{eqnarray*}

Для каждого $i \in [0,n-1]$ вычисляем следующее:
\begin{eqnarray*}
e_{1,i}^G &:=& \hg\left( C_i^G, C_i^H, a_{1,i}G - e_{0,i}^GC_i^G, b_{1,i}H - e_{0,i}^HC_i^H \right) \in \zp \\
e_{1,i}^H &:=& \hh\left( C_i^G, C_i^H, a_{1,i}G - e_{0,i}^GC_i^G, b_{1,i}H - e_{0,i}^HC_i^H \right) \in \zq \\
(e_{0,i}^G)' &:=& \hg\left( C_i^G, C_i^H, a_{0,i}G - e_{1,i}^G(C_i^G-G'), b_{0,i}H - e_{1,i}^H(C_i^H-H') \right) \in \zp \\
(e_{0,i}^H)' &:=& \hh\left( C_i^G, C_i^H, a_{0,i}G - e_{1,i}^G(C_i^G-G'), b_{0,i}H - e_{1,i}^H(C_i^H-H') \right) \in \zq
\end{eqnarray*}
Проверяем соответствие $(e_{0,i}^G)' = e_{0,i}^G$ и $(e_{0,i}^H)' = e_{0,i}^H$ в кортеже доказательства.

Если все проверки будут успешными, верификатор примет доказательство. В противном случае доказательство будет отклонено. Предполагается, что верификатор также проверяет каждый элемент кортежа доказательства, чтобы доказать его принадлежность к ожидаемой группе. Это делается, чтобы вычислить злоумышленника, которым может оказаться доказывающая сторона.
\end{document}
