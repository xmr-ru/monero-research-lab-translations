\documentclass[12pt,english]{mrl}
\usepackage[utf8x]{inputenc}
\usepackage[T2A]{fontenc}
\usepackage[english, russian]{babel}
\usepackage{color}
\renewcommand\refname{Ссылки}

\title{Функция ge\_fromfe\_frombytes\_vartime}
\authors{Шен Ноезер (Shen Noether)\footnote{\texttt{shen.noether@gmx.com}}}
\affiliations{Исследовательская лаборатория Monero (Monero Research Lab)}

\type{РЕЗЮМЕ}
\ident{UNPUBLISHED}

\begin{document}

\begin{center}
{\bfАннотация}
\end{center}

Мною рассматривается функция ge\_fromfe\_frombytes\_vartime, используемая с функциями образов ключей Monero.

\section{Вступление}
В этой короткой технической заметке мною рассматривается функция ge\_fromfe\_frombytes\_vartime, используемая в образах ключей Monero. Следует отметить, что этот код был унаследован от разработчиков оригинального протокола CryptoNote, которые, безусловно, являются специалистами в области криптографии, но чей недостаток заключается в неумении объяснять и комментировать свою работу. Также хотелось бы отметить, что прошлым летом мною уже была заменена большая часть криптографической библиотеки ed25519, используемой Monero, на вариант ref10, предложенный Бернштейном.

В недавно появившихся исследовательских работах (довольно известных авторов) рассматривалась возможность наложения случайной строки на точку на эллиптической кривой [см. \cite{Tib2010,Tib2013}]. Интересно, что функция «хеширования по точке», ge\_fromfe\_frombytes\_vartime, используемая протоколом CryptoNote \cite{CN}, кажется, не упоминалась ни в одной из этих работ, но потенциально является более эффективным алгоритмом.


\section{fe\_frombytes}
Очевидно, что эта часть является fe\_frombytes из ref10.

\section{Неизвестная часть}
Предположим, что сначала $y\equiv 0$, а sign $\equiv $ sign.

Следовательно, получаем:
\[
2u^2 + 1 - x \equiv 0
\]
что даёт нам $x\equiv 2u^2 + 1$.

Таким образом,
\[
2u^2 + 1 \equiv r_x^2 (w^2 - 2A^2 u^2)
\]
что даёт нам
\[
r_x = \left(\frac{2u^2 + 1}{w^2 - 2A^2 u^2}\right)^{\frac{1}{2}}.
\]

В этом случае мы правильно вычисляем квадратный корень с первой попытки. Теперь нам необходимо убедиться в том, что $y$ и $x$ находятся на эллиптической кривой.


\[
x_{p}=w^{2}-2A^{2}u^{2}=\left(2u^{2}+1\right)^{2}-2A^{2}u^{2}
\]
\[
rxt=\left(w/x_{p}\right)^{.5}
\]


\[
x_{t}=rxt^{2}\left(w^{2}-2A^{2}u^{2}\right)\to\left(\frac{w}{w^{2}-2A^{2}u^{2}}\right)\left(w^{2}-2A^{2}u^{2}\right)\to w
\]
(если $rxt$ действительно является квадратным корнем).
\[
y=\left(2u^{2}+1-x_{t}\right)
\]
\[
rx=-u\left(2A\left(A+2\right)\frac{w}{x_{p}}\right)^{\frac{1}{2}}=-\left(2A\left(A+2\right)\frac{u^{2}w}{w^{2}-2A^{2}u^{2}}\right)^{\frac{1}{2}}
\]


\[
z=-2Au^{2}=-\left(w-1\right)A=\left(1-w\right)A
\]


(следует отметить, что $-z=2Au^{2},$ $zA=-2A^{2}u^{2}$

\[
ry=z-w
\]


\[
Y^{2}=\left(z-w\right)^{2}
\]


\[
rz=z+w
\]


\[
Z^{2}=\left(z+w\right)^{2}
\]


\[
r_{x-final}=\left(z+w\right)\left(2A\left(A+2\right)\frac{u^{2}w}{w^{2}+zA}\right)^{\frac{1}{2}}
\]


\[
X^{2}=Z^{2}\left(\left(A+2\right)\frac{2Au^{2}w}{w^{2}+zA}\right)
\]
\[
=Z^{2}\left(A+2\right)\frac{-zw}{w^{2}+Az}
\]


\[
d=-\frac{A-2}{A+2}
\]


проверяем, действительно ли
\[
-X^{2}Z^{2}+Y^{2}Z^{2}=\left(Z^{2}\right)^{2}+dX^{2}Y^{2}
\]
или, другими словами, что
\[
Z^{4}\left(A+2\right)\frac{zw}{w^{2}+Az}+Z^{2}\left(z-w\right)^{2}=Z^{4}+\left(A-2\right)Z^{2}\frac{zw}{w^{2}+Az}\left(z-w\right)^{2}
\]
сокращаем $Z^{2}$:
\[
\left(z+w\right)^{2}\left(A+2\right)\frac{zw}{w^{2}+Az}+\left(z-w\right)^{2}\overset{?}{=}\left(z+w\right)^{2}+\left(A-2\right)\frac{zw}{w^{2}+Az}\left(z-w\right)^{2}
\]
После этого умножаем на $w^{2}+Az$

\[
\left(z+w\right)^{2}\left(A+2\right)zw+\left(z-w\right)^{2}\left(w^{2}+Az\right)
\]
\[\overset{?}{=}\left(z+w\right)^{2}\left(w^{2}+Az\right)+\left(A-2\right)\left(zw\right)\left(z-w\right)^{2}
\]
После включения $z=\left(1-w\right)A$ при помощи компьютерной алгебраической системы, такой как Maxima, проверяем равенство двух сторон.
\par
Теперь у нас имеется несколько операторов «если» для различных случаев. В первом случае проверяется, был ли в результате вычислений действительно получен отрицательный квадратный корень. Если это не так, проверяется, не был ли вычислен квадратный корень для отрицательного начального значения. Наконец, отметив, что $p = 2^255 - 19\equiv 1\ mod\ 4$, так что $-1$ является невычетом, и если взять произведение невычетов, то получится вычет, мы умножаем нашу попытку на $-1$.

\bibliographystyle{alpha}
\bibliography{main}

\end{document}
